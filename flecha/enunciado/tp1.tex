\documentclass{article}
\usepackage[spanish]{babel}
\usepackage[utf8]{inputenc}
\usepackage{amssymb}
\usepackage{xcolor}
\usepackage{xspace}
\usepackage{tabularx}
\usepackage[margin=1.9cm]{geometry}
\usepackage{amsmath}
\usepackage{minted}

\colorlet{darkgreen}{green!60!black}
\newcommand{\TODO}[1]{\textcolor{red}{TODO: #1}}
\newcommand{\HS}{\hspace{.5cm}}
\newcommand{\flecha}{\textsf{Flecha}\xspace}
\newcommand{\tok}[1]{\textcolor{red}{{\texttt{$\langle$#1$\rangle$}}}}
\newcommand{\nonterm}[1]{\textcolor{blue}{{\em{$\langle$#1$\rangle$}}}}
\newcommand{\produccion}[2]{
  \item[]
  \begin{tabularx}{\textwidth}{lrlr}
  #1 & $\xrightarrow{\hspace{.5cm}}$ & #2
  \end{tabularx}\\
}
\newcommand{\EMPTY}{$\epsilon$}
\newcommand{\ALT}{
  \\ & $\mid$ &
}

% AST
\newcommand{\type}[1]{\textcolor{darkgreen}{\texttt{#1}}}
%\renewcommand{\ast}[1]{\textcolor{darkgreen}{\texttt{\underline{#1}}}}
%\newcommand{\instruction}[1]{\textcolor{darkgreen}{\texttt{\underline{#1}}}}
\newcommand{\typedecl}[2]{\noindent
  \begin{tabularx}{\textwidth}{lrlr}
  #1 & $=$ & #2
  \end{tabularx}\\
}
\newcommand{\datadecl}[2]{\noindent
  \begin{tabularx}{\textwidth}{lrp{13cm}r}
  #1 & $::=$ & #2
  \end{tabularx}\\
}
\newcommand{\fl}[1]{\mintinline{haskell}{#1}}
\newcommand{\astkw}[1]{\texttt{\textcolor{red}{"#1"}}}
\newcommand{\astpart}[2]{$\underbrace{\text{#2}}_{\text{#1}}$}

\begin{document}
\begin{center}
\textsc{{\small Parseo y Generaci\'on de C\'odigo -- $2^{\text{do}}$ semestre 2018}} \medskip \\ 
\textsc{{\small Licenciatura en Inform\'atica con Orientaci\'on en Desarrollo de Software}} \medskip \\
\textsc{{\small Universidad Nacional de Quilmes}} \medskip \\
{\Large Trabajo pr\'actico 1} \medskip \\
{\large {\bf Parser para \flecha}} \medskip \\
\bigskip
Fecha de entrega: 4 de octubre
\end{center}

\tableofcontents

\section{Introducci\'on}

Este TP consiste en implementar un analizador sintáctico
para el lenguaje de programación funcional \flecha.
El analizador sintáctico puede estar implementado en el lenguaje de su
preferencia, y se puede definir manualmente (por ejemplo, usando la técnica
de descenso recursivo) o a través un generador de parsers\footnote{Por ejemplo
\texttt{yacc} para C/C++, \texttt{ANTLR} para Java, \texttt{ply} para Python, \texttt{happy} para Haskell,
etc.}.
En el apéndice~A se encuentra definida la gramática formal del lenguaje \flecha.
Si usan un generador de parsers, probablemente tengan que adaptar la gramática
para eliminar ambigüedades.

El analizador sintáctico debe poder analizar programas válidos escritos en lenguaje
\flecha y construir un árbol de sintaxis abstracta~(AST).
Se debe programar también la funcionalidad necesaria para generar
el AST en formato \texttt{JSON}, como se detalla en el apéndice~B,
a los efectos de comprobar que su implementación coincida con la esperada.

{\bf Nota:} En este TP no se evalúan cuestiones de estilo,
pero consideren que el TP~2 será una extensión del TP~1,
por lo que es recomendable usar buenas prácticas para facilitar
su propia tarea en el futuro.

\section{Características del lenguaje \flecha}

En esta sección describimos informalmente las características del lenguaje \flecha.
\begin{itemize}
\item {\bf Tipos.}
      \flecha es un lenguaje {\em no tipado}.
      Los tipos de datos existen únicamente como propiedades de los
      valores en tiempo de ejecución. Esto es similar a lo que
      ocurre en lenguajes a veces llamados ``dinámicos'', como
      SmallTalk, Python o JavaScript.
\item {\bf Tipos de datos primitivos.}
      \flecha cuenta con dos tipos de datos primitivos:
      enteros y caracteres. Por ejemplo:
\begin{minted}{haskell}
1      2      3      0      65536                        -- enteros
'a'    'z'    'A'    'Z'    '0'    '9'    '_'    '\n'    -- caracteres
\end{minted}
\item {\bf Estructuras.}
      \flecha cuenta con el tipo de las {\em estructuras}.
      Una estructura se forma aplicando un constructor a una lista
      de argumentos. Un constructor es cualquier identificador empezado
      por mayúsculas. Por ejemplo, las siguientes son estructuras:
\begin{minted}{haskell}
True                                   -- sin argumentos
False                                  -- sin argumentos
HeladoDeFrambuesa                      -- sin argumentos
Just 'a'                               -- un argumento
MkCoordenada 3 5                       -- dos argumentos
Cons 1 (Cons 2 (Cons 3 Nil))           -- dos argumentos (recursiva)
\end{minted}
\item {\bf Strings.}
      \flecha permite escribir listas de caracteres usando una sintaxis
      abreviada en forma de {\em string}. Por ejemplo,
      \fl{"hola"} es una manera abreviada de escribir
      la siguiente estructura:
\begin{minted}{haskell}
Cons 'h' (Cons 'o' (Cons 'l' (Cons 'a' Nil)))
\end{minted}
\item {\bf Operaciones aritméticas, relacionales y lógicas.}
  \flecha soporta los siguientes operadores:
  \begin{enumerate}
  \item Operaciones {\bf aritméticas} con enteros: suma (\fl{+}),
        resta (\fl{-}), multiplicación (\fl{*}), división entera (\fl{/})
        y resto en la división entera (\!\texttt{\%}).
  \item Operaciones {\bf relacionales} para comparar por
        igualdad (\fl{==}), desigualdad (\fl{!=}),
        mayor o igual (\fl{>=}), menor o igual (\fl{<=}),
        mayor estricto (\fl{>}), menor estricto (\fl{<}).
  \item Operaciones {\bf lógicas} para
        la conjunción, es decir, el ``y'' lógico (\fl{&&}),
        la disyunción, es decir, el ``o'' lógico (\fl{||}),
        y la negación (\fl{!}).
  \end{enumerate}
  Por ejemplo:
  \begin{minted}{haskell}
    n * (n - 1) / 2
    0 <= x && x <= 9 || 0 <= y && y <= 9
  \end{minted}
\item {\bf Definición de constantes y funciones.}
  Un programa en \flecha es una secuencia de definiciones de constantes
  y funciones. Por ejemplo:
\begin{minted}{haskell}
def miNumeroFavorito = 7
def cuadrado x = x * x
def main = cuadrado (cuadrado miNumeroFavorito)
\end{minted}
\item {\bf Alternativa condicional.}
  \flecha permite escribir expresiones condicionales
  de la forma \fl{if c then x else y}.
  Dado que el \fl{if} es una expresión, siempre debe tener una
  rama \fl{else}.
  En este TP todavía no nos interesa la {\em semántica} de las
  operaciones, pero vale aclarar que la estructura \fl{True}
  se considera el valor verdadero y la estructura \fl{False} es el
  valor falso, mientras que cualquier otro valor provocará un error
  en tiempo de ejecución.
\begin{minted}{haskell}
if True  then 1 else 2        ==> 1
if False then 1 else 2        ==> 2
if 7     then 1 else 2        ==> (error)
if Nil   then 1 else 2        ==> (error)
\end{minted}
  La construcción \fl{if} puede incorporar cero, una o varias ramas
  \fl{elif}:
\begin{minted}{haskell}
def diaSemana n =
    if n == 1 then Lunes
  elif n == 2 then Martes
  elif n == 3 then Miercoles
  elif n == 4 then Jueves
  elif n == 5 then Viernes
  elif n == 6 then Sabado
              else Domingo
\end{minted}
\item {\bf Pattern matching.}
\flecha permite hacer pattern matching usando la construcción \fl{case}.
La construcción \fl{case} recibe un valor y tiene varias {\bf ramas}.
Cada rama verifica si el valor es una estructura armada usando un
constructor. Por ejemplo:
\begin{minted}{haskell}
def longitud lista =
  case lista
  | Nil       -> 0
  | Cons x xs -> 1 + longitud xs

longitud "abc"    ==> 3
longitud False    ==> (error)
longitud 7        ==> (error)
\end{minted}
Por simplicidad, supondremos que el {\em pattern matching}
no puede incluir constructores anidados, por ejemplo:
\begin{minted}{haskell} 
  -- Lo siguiente no está permitido:
  def tieneExactamenteUnElemento lista =
    case lista
    | Nil                -> False
    | Cons x Nil         -> True    -- Nil adentro de Cons
    | Cons x (Cons y ys) -> False   -- Cons adentro de Cons

  -- Lo siguiente sí está permitido:
  def tieneExactamenteUnElemento lista =
    case lista
    | Nil       -> False
    | Cons x xs -> (case xs
                    | Nil       -> True
                    | Cons y ys -> False)
\end{minted}
\item {\bf Declaraciones locales.}
  Se permite declarar constantes y funciones locales con \fl{let}.
\begin{minted}{haskell}
  def f x y =
    let z = x + y in
      Coordenada z z

  f 10 20  ==> Coordenada 30 30

  def g x =
    let h y = x + y in
      Coordenada (g 1) (g 2)

  g 10  ==> Coordenada 11 12
\end{minted}
\item {\bf Secuenciación.}
  A pesar de ser un lenguaje funcional, \flecha cuenta con efectos
  secundarios para hacer entrada/salida.
  Para ello resulta útil contar con el operador de secuenciación,
  escrito con un punto y coma (\fl{;}).
  Si escribimos \texttt{p1; p2} esto tiene el efecto de ejecutar
  \texttt{p1}, descartando su valor, y a continuación ejecutar
  \texttt{p2}. Por ejemplo:
\begin{minted}{haskell}
def main =
  print "hola\n";
  print "chau\n"
\end{minted}
El operador de secuenciación es solamente una abreviatura.
Internamente el analizador sintáctico convierte
la expresión \texttt{(p1; p2)}
en la expresión \fl{(let _ = p1 in p2)}.
\item {\bf Funciones.}
En \flecha las funciones también son datos.
Se permite también definir funciones anónimas con notación {\em lambda}.
Por ejemplo:
\begin{minted}{haskell}
  def twice f = \ x -> f (f x)

  twice (Cons 1) Nil     ==> Cons 1 (Cons 1 Nil)

  def map = \ f lista -> case lista
                         | Nil       -> Nil
                         | Cons x xs -> Cons (f x) (map f xs)

  map (\ x -> 2 * x) (Cons 1 (Cons 2 (Cons 3 Nil)))
\end{minted}
\end{itemize}

\section{Construcción del árbol de sintaxis abstracta}

Como resultado del análisis sintáctico, debe construirse un árbol
de sintaxis abstracta o AST.
La representación interna del AST puede ser la que prefieran,
pero deben implementar la funcionalidad necesaria para generar
un AST en formato JSON de acuerdo con lo que se especifica en el
apéndice~B.

\section{Pautas de entrega}

Para entregar el TP se debe enviar el c\'odigo fuente por e-mail
a la casilla \texttt{foones@gmail.com} hasta las 23:59:59 del d\'ia
estipulado para la entrega, incluyendo \texttt{[TP lds-est-parse]} en
el asunto y el nombre de los integrantes del grupo en el cuerpo
del e-mail. No es necesario hacer un informe sobre el TP, pero se espera
que el c\'odigo sea razonablemente legible.
Se debe incluir un README
indicando las dependencias y el mecanismo de ejecuci\'on
recomendado para que el programa provea la funcionalidad pedida.
Se recomienda probar el programa con el conjunto de tests provistos.

\appendix

\section{Gramática formal de \flecha}

\subsection{Convenciones léxicas}

En esta sección se detalla el funcionamiento del analizador léxico.
Escribimos en \tok{MAYÚSCULAS} el nombre del símbolo terminal o {\em token}
que usaremos formalmente en la gramática.

\subsubsection*{Blancos y comentarios}
  Se ignoran los caracteres en blanco, incluyendo
  espacios (\verb|' '|, caracter \texttt{0x20}),
  tabs (\verb|'\t'|, caracter \texttt{0x09}),
  saltos de línea (\verb|'\n'|, caracter \texttt{0x0a}),
  y retornos de carro (\verb|'\r'|, caracter \texttt{0x0d}).

  Se ignoran los comentarios.
  Un comentario puede empezar en cualquier punto
  del programa con una secuencia de dos guiones seguidos, es decir dos
  veces el símbolo ``menos'', (\verb|--|).
  Un comentario termina en la siguiente ocurrencia de un salto de línea
  (\verb|'\n'|).

\subsubsection*{Identificadores}

  Un {\bf identificador} es una secuencia no vacía de símbolos consecutivos
  que empiezan con un caracter alfabético
  y pueden incluir
  minúsculas (\verb|a..z|),
  mayúsculas (\verb|A..Z|),
  caracteres numéricos (\verb|0..9|)
  y guiones bajos (\verb|_|).
  Distinguimos dos tipos de identificadores:
  \begin{enumerate}
  \item Identificador que comienza en minúscula,
        es decir, de la forma
        \verb|[a-z][_a-zA-Z0-9]*|.\\
        Se utiliza para nombres de variables, constantes y funciones.\\
        El token correspondiente es \tok{LOWERID}.
  \item Identificador que comienza en mayúscula,
        es decir, de la forma
        \verb|[A-Z][_a-zA-Z0-9]*|.\\
        Se utiliza para nombres de constructores.\\
        El token correspondiente es \tok{UPPERID}.
  \end{enumerate}
  Los identificadores podrían tener longitud arbitrariamente larga,
  pero se acepta que la implementación se limite a identificadores de
  longitud hasta 1023.

\subsubsection*{Constantes numéricas}
  Una constante numérica es una secuencia no vacía de dígitos
  decimales (\verb|0..9|).
  Se utiliza para representar enteros (escritos en decimal).
  El token correspondiente es \tok{NUMBER}.

  Las constantes numéricas podrían representar enteros arbitrariamente
  grandes,
  pero se acepta que la implementación se limite a números entre
  0 y $2^{31}-1$, es decir, el máximo entero positivo representable
  usando una representación con signo de 32 bits.

\subsubsection*{Constantes de caracter}
  Una constante de caracter consta de un caracter delimitado por comillas
  simples (\verb|'|).
  Por ejemplo, \verb|'a'| y \verb|'9'| son constantes de caracter.
  El token correspondiente es \tok{CHAR}.
  Para poder representar algunos caracteres especiales como retornos de línea,
  y comillas simples se aceptan las siguientes seis secuencias de escape:
  \begin{center}
  \begin{tabular}{ll}
  \verb|'\''| & representa la comilla simple \verb|'| \\
  \verb|'\"'| & representa la comilla doble \verb|"| \\
  \verb|'\\'| & representa la contrabarra \verb|\| \\
  \verb|'\t'| & representa un tab \verb|\t|, caracter \texttt{0x09}\\
  \verb|'\n'| & representa un salto de línea \verb|\n|, caracter \texttt{0x0a}\\
  \verb|'\r'| & representa un retorno de carro \verb|\r|, caracter \texttt{0x0d}\\
  \end{tabular}
  \end{center}
\subsubsection*{Constantes de {\em string}}
  Una constante de {\em string} consta de una secuencia de caracteres
delimitados por comillas dobles (\verb|"|).
El token correspondiente es \tok{STRING}.
Para las constantes de {\em string} se aceptan las mismas
seis secuencias de escape que en el caso de las constantes de
caracter. Por ejemplo, \verb|"Hola\n"| representa un {\em string}
de cinco caracteres, el último de los cuales es un salto de línea.

Los strings podrían tener longitud arbitrariamente larga,
pero se acepta que la implementación se limite a escribir constantes
de string de longitud hasta 1023.

\subsubsection*{Palabras clave}
\begin{center}
\begin{tabular}{ll}
\hspace{-5cm}{\bf Definiciones} \\
\texttt{def}    & \tok{DEF}  \\
\hspace{-5cm}{\bf Alternativa condicional} \\
\texttt{if}     & \tok{IF}   \\
\texttt{then}   & \tok{THEN} \\
\texttt{elif}   & \tok{ELIF} \\
\texttt{else}   & \tok{ELSE} \\
\hspace{-5cm}{\bf {\em Pattern matching}} \\
\texttt{case}   & \tok{CASE} \\
\hspace{-5cm}{\bf Declaraciones locales} \\
\texttt{let}    & \tok{LET}  \\
\texttt{in}     & \tok{IN}   \\
\end{tabular}
\end{center}

\subsubsection*{Símbolos reservados}
\begin{center}
\begin{tabular}{lll@{\hspace{1cm}}l}
\hspace{-.5cm}{\bf Delimitadores} \\
{\small Igual} &
\texttt{=} & \tok{DEFEQ}     & {\small para definiciones (\fl{def x = ...} y \fl{let x = ...}).}\\
{\small Punto y coma} &
\texttt{;} & \tok{SEMICOLON} & {\small para la secuenciación (\texttt{p1; p2}).}\\
{\small Paréntesis izquierdo} &
\texttt{(} & \tok{LPAREN} & {\small para agrupar expresiones.}\\
{\small Paréntesis derecho} &
\texttt{)} & \tok{RPAREN} & {\small para agrupar expresiones.}\\
{\small Contrabarra} &
\verb|\| & \tok{LAMBDA} & {\small para definir funciones anónimas \fl{(\ x -> x)}.}\\
{\small Barra vertical} &
\texttt{|} & \tok{PIPE} & {\small para las ramas del \fl{case}.}\\
{\small Flecha} &
\texttt{->} & \tok{ARROW} & {\small para las funciones anónimas y las ramas del \fl{case}.}\\
\hspace{-.5cm}{\bf Operadores lógicos} \\
{\small Conjunción} & \texttt{\&\&} & \tok{AND} \\
{\small Disyunción} & \texttt{||}   & \tok{OR}  \\
{\small Negación}   & \texttt{!}    & \tok{NOT} \\
\hspace{-.5cm}{\bf Operadores relacionales} \\
{\small Igualdad}       & \texttt{==} & \tok{EQ} \\
{\small Desigualdad}    & \texttt{!=} & \tok{NE} \\
{\small Mayor o igual}  & \texttt{>=} & \tok{GE} \\
{\small Menor o igual}  & \texttt{<=} & \tok{LE} \\
{\small Mayor estricto} & \texttt{>}  & \tok{GT} \\
{\small Menor estricto} & \texttt{<}  & \tok{LT} \\
\hspace{-.5cm}{\bf Operadores aritméticos} \\
{\small Suma}           & \texttt{+} & \tok{PLUS} \\
{\small Resta}          & \texttt{-} & \tok{MINUS} \\
{\small Multiplicación} & \texttt{*} & \tok{TIMES} \\
{\small División}       & \texttt{/} & \tok{DIV} \\
{\small Resto}          & \verb|%|   & \tok{MOD} \\
\end{tabular}
\end{center}

\subsection{Sintaxis}

En esta sección se da una gramática independiente del contexto para
\flecha.
\begin{itemize}
\produccion{
  \nonterm{programa}
}{
  \EMPTY
  \ALT
  \nonterm{programa} \nonterm{definición}
}
\produccion{
  \nonterm{definición}
}{
  \tok{DEF} \tok{LOWERID} \nonterm{parámetros} \tok{DEFEQ} \nonterm{expresión}
}
\produccion{
  \nonterm{parámetros}
}{
  \EMPTY \ALT \tok{LOWERID} \nonterm{parámetros}
}
\produccion{
  \nonterm{expresión}
}{
  \nonterm{expresiónExterna}
\ALT
  \nonterm{expresiónExterna} \tok{SEMICOLON} \nonterm{expresión}
}
\produccion{
  \nonterm{expresiónExterna}
}{
  \nonterm{expresiónIf}
  \ALT
  \nonterm{expresiónCase}
  \ALT
  \nonterm{expresiónLet}
  \ALT
  \nonterm{expresiónLambda}
  \ALT
  \nonterm{expresiónInterna}
}
\produccion{
  \nonterm{expresiónIf}
}{
  \tok{IF}
  \nonterm{expresionInterna}
  \tok{THEN}
  \nonterm{expresionInterna}
  \nonterm{ramasElse}
}
\produccion{
  \nonterm{ramasElse}
}{
  \tok{ELIF} \nonterm{expresionInterna} \tok{THEN} \nonterm{expresionInterna}
             \nonterm{ramasElse}
  \ALT
  \tok{ELSE} \nonterm{expresionInterna}
}
\produccion{
  \nonterm{expresiónCase}
}{
  \tok{CASE} \nonterm{expresionInterna} \nonterm{ramasCase}
}
\produccion{
  \nonterm{ramasCase}
}{
  \EMPTY
  \ALT
  \nonterm{ramaCase} \nonterm{ramasCase}
}
\produccion{
  \nonterm{ramaCase}
}{
  \tok{PIPE} \tok{UPPERID} \nonterm{parámetros} \tok{ARROW} \nonterm{expresionInterna}
}
\produccion{
  \nonterm{expresiónLet}
}{
  \tok{LET} \tok{ID} \nonterm{parámetros} \tok{DEFEQ}
  \nonterm{expresiónInterna} \tok{IN} \nonterm{expresiónExterna}
}
\produccion{
  \nonterm{expresiónLambda}
}{
  \tok{LAMBDA} \nonterm{parámetros} \tok{ARROW} \nonterm{expresiónExterna}
}
\produccion{
  \nonterm{expresiónInterna}
}{
  \nonterm{expresiónAplicación}
  \ALT
  \nonterm{expresiónInterna} \nonterm{operadorBinario} \nonterm{expresiónInterna}
  \ALT
  \nonterm{operadorUnario} \nonterm{expresiónInterna}
}
\produccion{
  \nonterm{operadorBinario}
}{
       \tok{AND}
  \ALT \tok{OR}
  \ALT \tok{EQ}
  \ALT \tok{NE}
  \ALT \tok{GE}
  \ALT \tok{LE}
  \ALT \tok{GT}
  \ALT \tok{LT}
  \ALT \tok{PLUS}
  \ALT \tok{MINUS}
  \ALT \tok{TIMES}
  \ALT \tok{DIV}
  \ALT \tok{MOD}
}
\produccion{
  \nonterm{operadorUnario}
}{
  \tok{NOT}
  \ALT
  \tok{MINUS}
}
\produccion{
  \nonterm{expresiónAplicación}
}{
  \nonterm{expresiónAtómica}
  \ALT
  \nonterm{expresiónAplicación} \nonterm{expresiónAtómica}
}
\produccion{
  \nonterm{expresiónAtómica}
}{
  \tok{LOWERID}
  \ALT
  \tok{UPPERID}
  \ALT
  \tok{NUMBER}
  \ALT
  \tok{CHAR}
  \ALT
  \tok{STRING}
  \ALT
  \tok{LPAREN} \nonterm{expresión} \tok{RPAREN}
}
\end{itemize}

Aclaraciones:
\begin{itemize}
\item El operador de secuenciación (punto y coma) tiene menor precedencia
      que todas las demás construcciones. Por ejemplo, son equivalentes:
\begin{minted}{haskell}
  if a then b else c ; d
  (if a then b else c) ; d
\end{minted}
\item Le siguen en precedencia las {\em expresiones externas}, es decir,
      el \fl{if}, el \fl{case}, el \fl{let} y las funciones anónimas
      (lambda). Por ejemplo, son equivalentes:
\begin{minted}{haskell}
  if a then b else c + d
  if a then b else (c + d)
\end{minted}
\item La condición y las ramas del \fl{if} y del \fl{case} no pueden
      ser expresiones externas, salvo que estén entre paréntesis.
      Asimismo, el valor ligado en un \fl{let} no puede ser una expresión
      externa.
      Por ejemplo,
\begin{minted}{haskell}
  if x then \ x -> x   else y        -- error de sintaxis
  if x then (\ x -> x) else y        -- OK

  case dir
  | Izq -> if a then b else c      -- error de sintaxis

  case dir
  | Izq -> (if a then b else c)    -- OK

  let x = let y = 3 in y   in x      -- error de sintaxis
  let x = (let y = 3 in y) in x      -- OK
\end{minted}
\item El cuerpo del \fl{let} y el cuerpo de las funciones anónimas (lambda)
      pueden ser expresiones externas. Por ejemplo:
\begin{minted}{haskell}
  let x = 1 in
  let y = 2 in
    \ z -> if a then x else y + z   -- OK
\end{minted}
\item Todos los operadores tienen menor precedencia que la aplicación.
Por ejemplo, son equivalentes:
\begin{minted}{haskell}
  f x y + g z
  (f x y) + (g z)
\end{minted}
\end{itemize}

\subsection{Asociatividad y precedencia}

Los operadores binarios y unarios deben respetar la siguiente tabla,
ordenada de menor a mayor precedencia. Los elementos de cada fila tienen
la misma precedencia entre ellos, y mayor precedencia que los de las
filas anteriores.
Todos los operadores binarios son asociativos a izquierda.
\[
\begin{tabular}{|c|}
\hline
  \verb`||`
\\
\hline
  \verb|&&|
\\
\hline
  \verb|!| {\em (unario)}
\\
\hline
  \verb|==| \HS
  \verb|!=| \HS
  \verb|>=| \HS
  \verb|<=| \HS
  \verb|>| \HS
  \verb|<|
\\
\hline
  \verb|+| \HS \verb|-|
\\
\hline
  \verb|*|
\\
\hline
  \verb|/| \HS \verb|%|
\\
\hline
  \verb|-| {\em (unario)}
\\
\hline
\end{tabular}
\]
Por ejemplo, son equivalentes:
\begin{minted}{haskell}
  ! x == y && z > a * b + c * d
  (!(x == y)) && (z > ((a * b) + (c * d)))
\end{minted}

\section{Árbol de sintaxis abstracta}

El árbol de sintaxis abstracta en formato JSON tiene la siguiente estructura.
Usamos \type{ID} para denotar un identificador arbitrario (string)
y \type{NUM} para denotar un número arbitrario (entero).
\medskip

\datadecl{\type{Program}}{[
  \type{Definition},
  $\hdots$,
  \type{Definition}
]
\hfill {\small Lista de $n$ definiciones, con $n \geq 0$.}
}
\medskip

\datadecl{\type{Definition}}{
[\astkw{Def}, \type{ID}, \type{Expr}]
\hfill {\small Definición.}
}
\medskip

\datadecl{\type{Expr}}{
[\astkw{ExprVar}, \type{ID}]
  \hfill {\small Variable.}
\ALT
[\astkw{ExprConstructor}, \type{ID}]
  \hfill {\small Constructor.}
\ALT
[\astkw{ExprNumber}, \type{NUM}]
  \hfill {\small Constante numérica.}
\ALT
[\astkw{ExprChar}, \type{NUM}]
  \hfill {\small Constante de caracter.}
\ALT
[\astkw{ExprCase}, \type{Expr}, [\type{CaseBranch}, $\hdots$, \type{CaseBranch}]]
  \hfill {\small Case de $n$ ramas, con $n \geq 0$.}
\ALT
[\astkw{ExprLet}, \type{ID}, \type{Expr}, \type{Expr}]
  \hfill {\small Declaración local.}
\ALT
[\astkw{ExprLambda}, \type{ID}, \type{Expr}]
  \hfill {\small Función anónima.}
\ALT
[\astkw{ExprApply}, \type{Expr}, \type{Expr}]
  \hfill {\small Aplicación.}
}
\medskip

\datadecl{\type{CaseBranch}}{
  [\astkw{CaseBranch}, \type{ID}, [\type{ID}, $\hdots$, \type{ID}], \type{Expr}]
  \hfill {\small Rama del case de $n$ parámetros, con $n \geq 0$.}
}

\begin{itemize}
\item
  Todas las lambdas reciben un único parámetro.
  Las lambdas que reciben varios parámetros deben expresarse como
  muchas lambdas anidadas. Por ejemplo, son equivalentes:
  \begin{minted}{haskell}
  \ x y z -> x + y
  \ x -> (\ y -> (\ z -> x + y))
  \end{minted}
\item
  Las funciones definidas con \fl{def} y \fl{let} se deben expresar
  con lambdas. Por ejemplo, son equivalentes:
  \begin{minted}{haskell}
  def f x y = let g z = x * z in y + z
  def f = \ x y -> let g = (\ z -> x * z) in y + z
  \end{minted}
\item
  El operador de secuenciación debe expresarse en términos del \fl{let}.
  Por ejemplo, son equivalentes:
  \begin{minted}{haskell}
  def main = print "hola\n"; print "chau\n"
  def main = let _ = print "hola\n" in print "chau\n"
  \end{minted}
\item
  Las ramas \fl{elif} de un \fl{if} deben expresarse internamente
  como ifs anidados. Por ejemplo, son equivalentes:
  \begin{minted}{haskell}
  if a then 1 elif b then 2 elif c then 3 else 4
  if a then 1 else (if b then 2 else (if c then 3 else 4))
  \end{minted}
\item
  El \fl{if} debe expresarse internamente como un \fl{case}.
  Por ejemplo, son equivalentes:
  \begin{minted}{haskell}
  if a then b else c

  case a
  | True  -> b
  | False -> c
  \end{minted}
\item
  Los caracteres se expresan con su código en la codificación ASCII o UTF-8.
  Se puede asumir que la entrada cuenta exclusivamente con caracteres
  imprimibles (en el rango 32..127).
  Por ejemplo, el AST de la expresión \fl{'A'}
  es \verb|["ExprChar", 97]|.
\item
  Los strings se expresan como listas de caracteres construidas usando
  \fl{Cons} y \fl{Nil}. Por ejemplo, son equivalentes:
  \begin{minted}{haskell}
  "hola"
  Cons 'h' (Cons 'o' (Cons 'l' (Cons 'a' Nil)))
  \end{minted}
\item
  La aplicación es siempre binaria, usando currificación.
  Por ejemplo, son equivalentes:
  \begin{minted}{haskell}
  f (g 1) 2 (g 3)
  ((f (g 1)) 2) (g 3)
  \end{minted}
\item
  Los operadores se expresan como la aplicación de {\em variables} 
  con nombres especiales a sus argumentos.
  Por ejemplo:
  \begin{itemize}
  \item El AST de la expresión \verb|!4| es:
\begin{verbatim}
  ["ExprApply", ["ExprVar", "NOT"], ["ExprNumber", 4]]
\end{verbatim}
  \item El AST de la expresión \verb|x + y| es:
\begin{verbatim}
  ["ExprApply", ["ExprApply", ["ExprVar", "ADD"], ["ExprVar", "x"]], ["ExprVar", "y"]]
\end{verbatim}
  \end{itemize}
Los nombres de los demás operadores son los siguientes:
\begin{center}
\begin{tabular}{ll@{\hspace{2cm}}ll@{\hspace{2cm}}ll}
  \verb`||` & \astkw{OR}
&
  \verb|&&| & \astkw{AND}
&
  \verb|!| {\em (unario)} & \astkw{NOT}
\\
  \verb|==| & \astkw{EQ}
&
  \verb|!=| & \astkw{NE}
&
  \verb|>=| & \astkw{GE}
\\
  \verb|<=| & \astkw{LE}
&
  \verb|>| & \astkw{GT}
&
  \verb|<| & \astkw{LT}
\\
  \verb|+| & \astkw{ADD}
&
  \verb|-| & \astkw{SUB}
&
  \verb|*| & \astkw{MUL}
\\
  \verb|/| & \astkw{DIV}
&
  \verb|%| & \astkw{MOD}
&
  \verb|-| {\em (unario)} & \astkw{UMINUS}
\end{tabular}
\end{center}
Observar que no puede haber conflicto con los nombres de otras variables,
porque las variables siempre tienen nombres empezados en minúsculas.
\end{itemize}

\subsection{Ejemplo completo de análisis sintáctico}

\subsubsection*{Programa}

\begin{minted}{haskell}
def sumar lista = 
  case lista
  | Nil       -> 0
  | Cons x xs -> x + sumar xs

def main = sumar (Cons 1 (Cons 2 Nil))
\end{minted}

\subsubsection*{AST obtenido}

\begin{verbatim}
[
 ["Def", "sumar",
  ["ExprLambda", "lista",
   ["ExprCase",
    ["ExprVar", "lista"],
    [
     ["CaseBranch", "Nil", [],
      ["ExprNumber", 0]
     ],
     ["CaseBranch", "Cons", ["x", "xs"],
      ["ExprApply",
       ["ExprApply",
        ["ExprVar", "ADD"],
        ["ExprVar", "x"]
       ],
       ["ExprApply",
        ["ExprVar", "sumar"],
        ["ExprVar", "xs"]
       ]
      ]
     ]
    ]
   ]
  ]
 ],
 ["Def", "main",
  ["ExprApply",
   ["ExprVar", "sumar"],
   ["ExprApply",
    ["ExprApply",
     ["ExprConstructor", "Cons"],
     ["ExprNumber", 1]
    ],
    ["ExprApply",
     ["ExprApply",
      ["ExprConstructor", "Cons"],
      ["ExprNumber", 2]
     ],
     ["ExprConstructor", "Nil"]
    ]
   ]
  ]
 ]
]
\end{verbatim}



\end{document}

